\documentclass[a4paper, 12pt]{article}
\usepackage[slovene]{babel}
\usepackage[utf8]{inputenc}
\usepackage[T1]{fontenc}
\usepackage{lmodern}
\usepackage{amsmath, amssymb}


\newtheorem{izrek}{Izrek}[section]

%\theoremstyle{definition}
\newtheorem{definicija}{Definicija}[section]

\newtheorem{opomba}{Opomba}[section]

\newtheorem{lema}{Lema}[section]

\newtheorem{trditev}{Trditev}[section]




\title{
    Distance vector of non tubical nanotube fullerenes of type-(5-0)
}

\author{Amanda Babič, Aljaž Flus \\
        {\small Mentorja Riste Škrekovski, Janoš Vidali}}
\date{7. 3. 2025}


\begin{document}

\maketitle

\newpage

\section{Uvod}

\begin{definicija}
    Vektor razdalje $d_{u}$ je vektor, katerega $i$-ta koordinata predstavlja število vozlišč, ki so od izbranega vozlišča $u$ oddaljeni natanko za $i$. 
\end{definicija}

\begin{definicija}
    Graf fulerena je 3-povezan, 3-regularen ravninski graf, sestavljen izključno iz petkotnih in šestkotnih ploskev.
\end{definicija}

\begin{opomba}
    Po Eulerjevi formuli je število petkotnih ploskev vedno 12.
\end{opomba}

\begin{definicija}
    Z $L_{0}$ začetni sloj kot množico vozličš, ki so sosednji s petkotnikom $p$, ki je središčni petkornik na začetku našega nanotuba 
    in definiramo s $F_0 = \{p\}$. Za vsak $j = 1, \dots , k$ množica plasti $F_j$ vsebuje vse plasti, ki so sosednje z vozlišči iz $L_{j-1}$ 
    in niso v $F_{j-1}$. Podobno $L_j$ vsebuje vsa vozlišča, ki so sosednja plasti is $F_j$ in niso v $L_{j-1}$. Zato je nanotub sestavljen Iz 
    $k+1$ plasti, kjer $L_0$ in $L_{k}$ vsebujeta $5$ vozličš, vsaka vmesna plast pa vsebuje $10$ vozlišč.
\end{definicija}

\begin{definicija}
    Naj bo $e = uv$ povezava v grafu $C_{10k}$, kjer je $u \in L_{j-1}$ in $v \in L_j$. Vozlišču u rečemo odhodno vozlišče za $L_{j-1}$, 
    vozlišču $v$ pa pravimo da je prihodno vozlišče za $L_j$. Tukaj lahko opazimo, da imamo v vsaki plasti $5$ odhodnih in $5$ prihodnid vozlišč,
    ki se zaporedoma izmenjujejo.
\end{definicija}

\begin{definicija}
    Naj bo $G$ neprazen končen povezan graf in $v$ vozlišče v grafu $G$. Distančna particija $\pi_{d}(v)$ 
    relativno na $v$ je skupina disjunktnih množic:
    \begin{itemize}
        \item $D_{0} = {v},$
        \item $D_{i} = {u : d(v,u) = i}, i= 1,2,3, \dots , ecc(v),$ \\
        kjer je $ecc(v)$ ekscentričnost vozlišča $v$, t.j. $ecc(v) = \max\limits_{u \in V(G)} d(v,u)$

    \end{itemize}
\end{definicija}

\begin{definicija}
    Naj bo $G$ neprazen končen povezan graf in $v$ vozlišče v $G$. Vektor distančne particije $DV(v) \in \mathbb{N}^{diam(G)}$ za vozlišče $v$
    definiramo kot
    $$
    DV(v) = (n_{0}(v), n_{1}(v), \dots , n_{diam(G)}(v)),
    $$
    kjer je $n_{i}(v) = |D_{i}|$ za $i=0,1, \dots , ecc(v)$, in $n_{i}(v) = 0$ za $ecc(v) < i \leq diam(G)$.
\end{definicija}

V naslednjem delu poročila bomo zaradi večje preglednosti izpustili ničelne komponente vektorja $DV(v)$. Pripomnemo lahko še 
da bo $n_{0}(v) = 1$ za vsak $v \in V(G)$

\section{Preverjanje izrekov}

\begin{lema}
    Če imamo 
    $$
    diam(C_{10k}) = \begin{cases}
                    2k+1, k=2;\\
                    2k, k \in \{3,4\};\\
                    2k-1, k \geq 5.

                    \end{cases}
    $$
    Potem lahko izračunamo ekscentričnosti. \\
    V kodi sva preverila to lemo za zečetnih 20 $k$ in so rezultati pravilni. Iz tega sklepamo da to velja za vse $k \in \mathbb{N}$
\end{lema}

\begin{lema}
    Za ekscentričnosti vozlišč v grafu $C_{10k}$ imamo:
    \begin{itemize}
        \item Če je $k=2$, potem je $ecc(v) = 5$ ta vse $v \in V(C_{10k})$.
        \item Če je $k=3$, potem je $ecc(v) = 6$ ta vse $v \in V(C_{10k})$.
        \item Če je $k \geq 4$ in $v \in L_{j}^{in} \cup L_{l-j}^{out}$ za $1 \leq j \leq \lfloor k/2 \rfloor$ potem je $$ecc(v) = 2(k-j) + \delta,$$
              kjer je $\delta = 2 $ za $(k,j) = (4,2)$, $\delta = 1$ za $(k , j) \in \{(4,1), (5,2), (6,3) \}$ in $\delta = 0$ sicer.
        \item Če je $k \geq 4$ in $v \in L_{j}^{out} \cup L_{k-j}^{in}$ za $0 \leq j \leq \lfloor k/2 \rfloor$ potem je $$ecc(v) = 2(k-j) - 1 +\delta$$
        kjer je $\delta = 2 $ za $(k,j) \in \{(4,1), (5,2)\}$, $\delta = 1$ za $(k , j) \in \{(4,0), (5,1), (6,2), (7,3) \}$ in $\delta = 0$ sicer.
    \end{itemize}
\end{lema}

Za $k \geq 2$ in za vsak $j=0,1, \dots , k$ plast $L_j$ razdeli naš nanotub na dva disjuntna dela. Lev del sestavljajo plasti $L_i$ za $i = 0,1,\dots , j-1$, desni
 del pa je sestavljen iz plasti $L_i$ za $I = j+1, \dots, k$. Z $L(v)$ označimo levo stran particije, z $R(v)$ pa desno stran. Z $D(v)$ pa označimo distančni vektor znotraj plasti $L_j$.

\begin{trditev}
    Naj bo $k \geq 2$ in naj bo $v$ vozlišče iz grafa $C_{10k}$ tak da velja $v \in L_j, 0 \leq j \leq k$. Potem velja
    $$
    D(v) = \begin{cases}
            (1,2,2), \text{če } j \in \{0, k \}, \\
            (1,2,2,2,2,1), \text{sicer}, 
            \end{cases}
    $$
    in 
    $$
            DV(v) = L(v) + D(v) + R(v).
    $$

\end{trditev}

Če imamo $u \in L_j^{in}$ in $v \in L_{k-j}^{out}$ zaradi simetrije velja $R(u) = L(v)$. Velja tudi obratno, če imamo $u \in L_j^{out}$ in 
$v \in L_{k-j}^{in}$ prav tako velja $R(u) = L(v)$. Zaradi tega je dovlj izračunati samo $L(v)$. Izračuni so objavljeni v nasledni tabeli.

$\begin{array}{ |c|c|c| }
    \hline
    L(v) & v \in L_j^{in} & v \in L_j^{out}  \\
    \hline
    j = 1 & [0, 1, 2, 2, 0, 0] & [0, 0, 2, 2, 1, 0] \\
    \hline
    j = 2, k = 2 & [0, 1, 4, 6, 3, 1] &  \\
    \hline
    j = 2, k \geq 3 & [0, 1, 2, 4, 4, 3, 1] & [0, 0, 2, 3, 5, 4, 1] \\
    \hline
    j = 3, k = 3 & [0, 1, 4, 6, 6, 6, 2] & \\
    \hline
    j = 3, k \geq 4 & [0, 1, 2, 4, 5, 7, 5, 1] & [0, 0, 2, 3, 5, 6, 7, 2] \\
    \hline
    j = 4, k = 4 & [0, 1, 4, 6, 6, 6, 6, 5, 1] & \\
    \hline
    j = 4, k \geq 5 & [0, 1, 2, 4, 5, 7, 7, 7, 2] & [0, 0, 2, 3, 5, 6, 7, 6, 6] \\
    \hline
    j = 5, k = 5 & [0, 1, 4, 6, 6, 6, 6, 5, 6, 5] & \\
    \hline
    j = 5, k \geq 6 & [0, 1, 2, 4, 5, 7, 7, 7, 6, 6] & [0, 0, 2, 3, 5, 6, 7, 6, 6, 5, 5]\\
    \hline
    6 \leq j \leq k-1, k \geq 7 & [0, 1, 2, 4, 5, 7, 7, 7, 6, 6, 5^{\#2(j-5)}] & [0, 0, 2, 3, 5, 6, 7, 6, 6, 5^{\#2(j-4)}]\\
    \hline
    j = k, k \geq 7 & [0, 1, 4, 6, 6, 6, 6, 5, 6, 5^{\#(2k - 9)}] & \\
    \hline

\end{array}$

\begin{izrek}
Naj bo $k \geq 10$. Dodatno naj bo $x = \text{'in'}$, če je $k$ sod in $x = \text{'out'}$, če je $k$ lih. Tako lahko izračunamo vektorje distančne 
particije za vsa vozlišča $C_{10k}$. Te vektorji so napisani v naslednji tabeli.
\end{izrek}

$\begin{array}{|c|c|}
    \hline

    & \text{Vektor distanc DV(v)}\\
    \hline    
    j = 0  \text{ in } v \in L_j^{out} & [1, 3, 6, 6, 6, 6, 6, 5, 6, 5^{\#(2k - 9)}] \\
    \hline
    j = 1  \text{ in } v \in L_j^{in}  & [1, 3, 6, 7, 7, 7, 7, 6, 6, 5^{\#(2k - 10)}]        \\
    \hline
    j = 1  \text{ in } v \in L_j^{out}  &  [1, 3, 6, 8, 8, 8, 7, 7, 6, 6, 5^{\#(2k - 12)}]       \\
    \hline
    j = 2  \text{ in } v \in L_j^{in}  &  [1, 3, 6, 9, 11, 10, 8, 6, 6, 5^{\#(2k - 12)}]       \\
    \hline
    j = 2  \text{ in } v \in L_j^{out}  &  [1, 3, 6, 9, 12, 12, 8, 7, 6, 6, 5^{\#(2k - 14)}]       \\
    \hline
    j = 3  \text{ in } v \in L_j^{in}  &  [1, 3, 6, 9, 12, 14, 12, 7, 6, 5^{\#(2k - 14)}]       \\
    \hline
    j = 3  \text{ in } v \in L_j^{out}  & [1, 3, 6, 9, 12, 14, 14, 9, 6, 6, 5^{\#(2k - 16)}]        \\
    \hline
    j = 4  \text{ in } v \in L_j^{in}  &  [1, 3, 6, 9, 12, 14, 14, 13, 8, 5^{\#(2k - 16)}]       \\
    \hline
    j = 4  \text{ in } v \in L_j^{out}  &  [1, 3, 6, 9, 12, 14, 14, 13, 12, 6, 5^{\#(2k - 18)}]       \\
    \hline
    j = 5, k>12, \text{sod }  \text{in } v \in L_j^{in}  & [1, 3, 6, 9, 12, 14, 14, 13, 12, 11, 10, 5^{\#(2k - 21)}]  \\
    \hline
    j = 5, k>12, \text{sod }  \text{in } v \in L_j^{out}  & [1, 3, 6, 9, 12, 14, 14, 13, 12, 11, 5^{\#(2k - 19)}]  \\
    \hline
    j = 5, k>13, \text{lih }  \text{in } v \in L_j^{in} &  [1, 3, 6, 9, 12, 14, 14, 13, 12, 11, 10, 5^{\#(2k - 23)}] \\
    \hline
    j = 5, k>13, \text{lih }  \text{in } v \in L_j^{out} & [1, 3, 6, 9, 12, 14, 14, 13, 12, 11, 5^{\#(2k - 21)}] \\
    \hline
    j > 5, k>14, \text{sod }  \text{in } v \in L_j^{out}  & [1, 3, 6, 9, 12, 14, 14, 13, 12, 11, 10^{\#(2j - 10)}, 5^{\#(2k - 4j + 1)}]  \\
    \hline
    j > 5, k>14, \text{sod }  \text{in } v \in L_j^{in}  & [1, 3, 6, 9, 12, 14, 14, 13, 12, 11, 10^{\#(2j - 9)}, 5^{\#(2k - 4j - 1)}] \\
    \hline
    j > 5, k>15, \text{lih }  \text{in } v \in L_j^{out} & [1, 3, 6, 9, 12, 14, 14, 13, 12, 11, 10^{\#(2j - 9)}, 5^{\#(2k - 4j - 3)}] \\
    \hline
    j > 5, k>15, \text{lih }  \text{in } v \in L_j^{in} & [1, 3, 6, 9, 12, 14, 14, 13, 12, 11, 10^{\#(2j - 10)}, 5^{\#(2k - 4j - 1)}] \\
    \hline
    j = \lfloor k/2 \lfloor, k \text{ sod }  \text{ za vsak } v \in L_j  & [1, 3, 6, 9, 12, 14, 14, 13, 12, 11, 10^{\#(k - 10)}, 5]        \\
    \hline
    j = \lfloor k/2 \lfloor, k \text{ lih }  \text{ za vsak } v \in L_j  & [1, 3, 6, 9, 12, 14, 14, 13, 12, 11, 10^{\#(k - 11)}, 5]        \\
    \hline
\end{array}$


\end{document}