\documentclass[a4paper, 12pt]{article}
\usepackage[slovene]{babel}
\usepackage[utf8]{inputenc}
\usepackage[T1]{fontenc}
\usepackage{lmodern}
\usepackage{amsmath, amssymb}



\newtheorem{izrek}{Izrek}[section]

%\theoremstyle{definition}
\newtheorem{definicija}{Definicija}[section]

\newtheorem{opomba}{Opomba}[section]

\newtheorem{lema}{Lema}[section]

\newtheorem{trditev}{Trditev}[section]




\title{
    Distance vector of non tubical nanotube fullerenes of type-(5-0)
}

\author{Amanda Babič, Aljaž Flus \\
        {\small Mentorja Riste Škrekovski, Janoš Vidali}}
\date{\today}


\begin{document}

\maketitle

\newpage

\section{Uvod}

\begin{definicija}
    Vektor razdalje $d_{u}$ je vektor, katerega $i$-ta koordinata predstavlja število vozlišč, ki so od izbranega vozlišča $u$ oddaljeni natanko za $i$. 
\end{definicija}

\begin{definicija}
    Graf fulerena je 3-povezan, 3-regularen ravninski graf, sestavljen izključno iz petkotnih in šestkotnih ploskev.
\end{definicija}

\begin{opomba}
    Po Eulerjevi formuli je število petkotnih ploskev vedno 12.
\end{opomba}

\begin{definicija}
    Z $L_{0}$ začetni sloj kot množico vozličš, ki so sosednji s petkotnikom $p$, ki je središčni petkornik na začetku našega nanotuba 
    in definiramo s $F_0 = \{p\}$. Za vsak $j = 1, \dots , k$ množica plasti $F_j$ vsebuje vse plasti, ki so sosednje z vozlišči iz $L_{j-1}$ 
    in niso v $F_{j-1}$. Podobno $L_j$ vsebuje vsa vozlišča, ki so sosednja plasti is $F_j$ in niso v $L_{j-1}$. Zato je nanotub sestavljen Iz 
    $k+1$ plasti, kjer $L_0$ in $L_{k}$ vsebujeta $5$ vozličš, vsaka vmesna plast pa vsebuje $10$ vozlišč.
\end{definicija}

\begin{definicija}
    Naj bo $e = uv$ povezava v grafu $C_{10k}$, kjer je $u \in L_{j-1}$ in $v \in L_j$. Vozlišču u rečemo odhodno vozlišče za $L_{j-1}$, 
    vozlišču $v$ pa pravimo da je prihodno vozlišče za $L_j$. Tukaj lahko opazimo, da imamo v vsaki plasti $5$ odhodnih in $5$ prihodnid vozlišč,
    ki se zaporedoma izmenjujejo.
\end{definicija}

\begin{definicija}
    Naj bo $G$ neprazen končen povezan graf in $v$ vozlišče v grafu $G$. Distančna particija $\pi_{d}(v)$ 
    relativno na $v$ je skupina disjunktnih množic:
    \begin{itemize}
        \item $D_{0} = {v},$
        \item $D_{i} = {u : d(v,u) = i}, i= 1,2,3, \dots , ecc(v),$ \\
        kjer je $ecc(v)$ ekscentričnost vozlišča $v$, t.j. $ecc(v) = \max\limits_{u \in V(G)} d(v,u)$

    \end{itemize}
\end{definicija}

\begin{definicija}
    Naj bo $G$ neprazen končen povezan graf in $v$ vozlišče v $G$. Vektor distančne particije $DV(v) \in \mathbb{N}^{diam(G)}$ za vozlišče $v$
    definiramo kot
    $$
    DV(v) = (n_{0}(v), n_{1}(v), \dots , n_{diam(G)}(v)),
    $$
    kjer je $n_{i}(v) = |D_{i}|$ za $i=0,1, \dots , ecc(v)$, in $n_{i}(v) = 0$ za $ecc(v) < i \leq diam(G)$.
\end{definicija}

V naslednjem delu poročila bomo zaradi večje preglednosti izpustili ničelne komponente vektorja $DV(v)$. Pripomnemo lahko še 
da bo $n_{0}(v) = 1$ za vsak $v \in V(G)$

\section{Preverjanje izrekov}

\begin{lema}
    Če imamo 
    $$
    diam(C_{10k}) = \begin{cases}
                    2k+1, k=2;\\
                    2k, k \in \{3,4\};\\
                    2k-1, k \geq 5.

                    \end{cases}
    $$
    Potem lahko izračunamo ekscentričnosti. \\
    V kodi sva preverila to lemo za zečetnih 20 $k$ in so rezultati pravilni. Iz tega sklepamo da to velja za vse $k \in \mathbb{N}$
\end{lema}

\begin{lema}
    Za ekscentričnosti vozlišč v grafu $C_{10k}$ imamo:
    \begin{itemize}
        \item Če je $k=2$, potem je $ecc(v) = 5$ ta vse $v \in V(C_{10k})$.
        \item Če je $k=3$, potem je $ecc(v) = 6$ ta vse $v \in V(C_{10k})$.
        \item Če je $k \geq 4$ in $v \in L_{j}^{in} \cup L_{l-j}^{out}$ za $1 \leq j \leq \lfloor k/2 \rfloor$ potem je $$ecc(v) = 2(k-j) + \delta,$$
              kjer je $\delta = 2 $ za $(k,j) = (4,2)$, $\delta = 1$ za $(k , j) \in \{(4,1), (5,2), (6,3) \}$ in $\delta = 0$ sicer.
        \item Če je $k \geq 4$ in $v \in L_{j}^{out} \cup L_{k-j}^{in}$ za $0 \leq j \leq \lfloor k/2 \rfloor$ potem je $$ecc(v) = 2(k-j) - 1 +\delta$$
        kjer je $\delta = 2 $ za $(k,j) \in \{(4,1), (5,2)\}$, $\delta = 1$ za $(k , j) \in \{(4,0), (5,1), (6,2), (7,3) \}$ in $\delta = 0$ sicer.
    \end{itemize}
\end{lema}

Lemo 2 sva preverila in ugotovila, da velja za $k=2$ in $k=3$. Vendar pa se pri večjih vrednostih $k$ izkaže, da lema ne drži, pri čemer ni očitne rešitve, kako bi jo ustrezno popravila.

Za $k \geq 2$ in za vsak $j=0,1, \dots , k$ plast $L_j$ razdeli naš nanotub na dva disjuntna dela. Lev del sestavljajo plasti $L_i$ za $i = 0,1,\dots , j-1$, desni
 del pa je sestavljen iz plasti $L_i$ za $I = j+1, \dots, k$. Z $L(v)$ označimo levo stran particije, z $R(v)$ pa desno stran. Z $D(v)$ pa označimo distančni vektor znotraj plasti $L_j$.

\begin{trditev}
    Naj bo $k \geq 2$ in naj bo $v$ vozlišče iz grafa $C_{10k}$ tak da velja $v \in L_j, 0 \leq j \leq k$. Potem velja
    $$
    D(v) = \begin{cases}
            (1,2,2), \text{če } j \in \{0, k \}, \\
            (1,2,2,2,2,1), \text{sicer}, 
            \end{cases}
    $$
    in 
    $$
            DV(v) = L(v) + D(v) + R(v).
    $$

\end{trditev}

Če imamo $u \in L_j^{in}$ in $v \in L_{k-j}^{out}$ zaradi simetrije velja $R(u) = L(v)$. Velja tudi obratno, če imamo $u \in L_j^{out}$ in 
$v \in L_{k-j}^{in}$ prav tako velja $R(u) = L(v)$. Zaradi tega je dovlj izračunati samo $L(v)$. Izračuni so objavljeni v Tabela 1.

\begin{table}
\centering
$\begin{array}{ |c|c|c| }
    \hline
    L(v) & v \in L_j^{in} & v \in L_j^{out}  \\
    \hline
    j = 1 & [0, 1, 2, 2, 0, 0] & [0, 0, 2, 2, 1, 0] \\
    \hline
    j = 2, k = 2 & [0, 1, 4, 6, 3, 1] &  \\
    \hline
    j = 2, k \geq 3 & [0, 1, 2, 4, 4, 3, 1] & [0, 0, 2, 3, 5, 4, 1] \\
    \hline
    j = 3, k = 3 & [0, 1, 4, 6, 6, 6, 2] & \\
    \hline
    j = 3, k \geq 4 & [0, 1, 2, 4, 5, 7, 5, 1] & [0, 0, 2, 3, 5, 6, 7, 2] \\
    \hline
    j = 4, k = 4 & [0, 1, 4, 6, 6, 6, 6, 5, 1] & \\
    \hline
    j = 4, k \geq 5 & [0, 1, 2, 4, 5, 7, 7, 7, 2] & [0, 0, 2, 3, 5, 6, 7, 6, 6] \\
    \hline
    j = 5, k = 5 & [0, 1, 4, 6, 6, 6, 6, 5, 6, 5] & \\
    \hline
    j = 5, k \geq 6 & [0, 1, 2, 4, 5, 7, 7, 7, 6, 6] & [0, 0, 2, 3, 5, 6, 7, 6, 6, 5, 5]\\
    \hline
    6 \leq j \leq k-1, k \geq 7 & [0, 1, 2, 4, 5, 7, 7, 7, 6, 6, 5^{\#2(j-5)}] & [0, 0, 2, 3, 5, 6, 7, 6, 6, 5^{\#2(j-4)}]\\
    \hline
    j = k, k \geq 7 & [0, 1, 4, 6, 6, 6, 6, 5, 6, 5^{\#(2k - 9)}] & \\
    \hline
\end{array}$
\caption{Vektorji distančne particije, kjer gledamo samo vektorje ki so levo od našega vozlišča. Notacija $\#5^k$ nam predstavlja število petic zapored na koncu vektorja.}
\end{table}

V Tabela 1 opazimo, da se distančni vektorji razlikujejo za vozlišča znotraj iste orbite. To je posledica dejstva, da so vozlišča lahko bodisi vhodna bodisi izhodna. Vidimo, da je prvo število v vseh vektorjih vedno enako $0$, 
saj upoštevamo le vozlišča v orbitah na levi strani izbranega vozlišča. Pri izhodnih vozliščih opazimo, da je tudi drugi element enak $0$, kar pomeni, da sta potrebni vsaj dve potezi, da zapustimo svojo orbito. Poleg tega opazimo, 
da se distančni vektorji od $k=7$ naprej stabilizirajo.


\begin{izrek}
Naj bo $k \geq 10$. Dodatno naj bo $x = \text{'in'}$, če je $k$ sod in $x = \text{'out'}$, če je $k$ lih. Tako lahko izračunamo vektorje distančne 
particije za vsa vozlišča $C_{10k}$. Te vektorji so napisani v Tabela 2.
\end{izrek}

\begin{table}
    
$\begin{array}{|c|c|}
    \hline

    & \text{Vektor distanc DV(v)}\\
    \hline    
    j = 0  \text{ in } v \in L_j^{out} & [1, 3, 6, 6, 6, 6, 6, 5, 6, 5^{\#(2k - 9)}] \\
    \hline
    j = 1, k \text{ sod }  \text{ in } v \in L_j^{in}  & [1, 3, 6, 7, 7, 7, 7, 6, 6, 5^{\#(2k - 12)}]        \\
    \hline
    j = 1, k \text{ sod }  \text{ in } v \in L_j^{out}  &  [1, 3, 6, 8, 8, 8, 7, 7, 6, 6, 5^{\#(2k - 10)}]       \\
    \hline
    j = 1, k \text{ lih }  \text{ in } v \in L_j^{in}  & [1, 3, 6, 8, 8, 8, 7, 7, 6, 6, 5^{\#(2k - 14)}]        \\
    \hline
    j = 1, k \text{ lih }  \text{ in } v \in L_j^{out}  &  [1, 3, 6, 7, 7, 7, 7, 6, 6, 5^{\#(2k - 12)}]       \\
    \hline
    j = 2, k \text{ sod }  \text{ in } v \in L_j^{in}  &  [1, 3, 6, 9, 11, 10, 8, 6, 6, 5^{\#(2k - 12)}]       \\
    \hline
    j = 2, k \text{ sod }  \text{ in } v \in L_j^{out}  &  [1, 3, 6, 9, 12, 12, 8, 7, 6, 6, 5^{\#(2k - 14)}]       \\
    \hline
    j = 2, k \text{ lih }  \text{ in } v \in L_j^{in}  &  [1, 3, 6, 9, 11, 10, 8, 6, 6, 5^{\#(2k - 14)}]       \\
    \hline
    j = 2, k \text{ lih }  \text{ in } v \in L_j^{out}  &  [1, 3, 6, 9, 12, 12, 8, 7, 6, 6, 5^{\#(2k - 16)}]       \\
    \hline
    j = 3, k \text{ sod }  \text{ in } v \in L_j^{in}  &  [1, 3, 6, 9, 12, 14, 14, 9, 6, 6, 5^{\#(2k - 16)}]       \\
    \hline
    j = 3 , k \text{ sod } \text{ in } v \in L_j^{out}  & [1, 3, 6, 9, 12, 14, 12, 7, 6, 5^{\#(2k - 14)}]        \\
    \hline
    j = 3 , k \text{ lih }  \text{ in } v \in L_j^{in}  &  [1, 3, 6, 9, 12, 14, 14, 9, 6, 6, 5^{\#(2k - 18)}]       \\
    \hline
    j = 3 , k \text{ lih }  \text{ in } v \in L_j^{out}  & [1, 3, 6, 9, 12, 14, 12, 7, 6, 5^{\#(2k - 16)}]        \\
    \hline
    j = 4 , k \text{ sod }  \text{ in } v \in L_j^{in}  &  [1, 3, 6, 9, 12, 14, 14, 13, 8, 5^{\#(2k - 16)}]       \\
    \hline
    j = 4 , k \text{ sod }  \text{ in } v \in L_j^{out}  &  [1, 3, 6, 9, 12, 14, 14, 13, 12, 6, 5^{\#(2k - 18)}]       \\
    \hline
    j = 4 , k \text{ lih }  \text{ in } v \in L_j^{in}  &  [1, 3, 6, 9, 12, 14, 14, 13, 8, 5^{\#(2k - 18)}]       \\
    \hline
    j = 4 , k \text{ lih }  \text{ in } v \in L_j^{out}  &  [1, 3, 6, 9, 12, 14, 14, 13, 12, 6, 5^{\#(2k - 20)}]       \\
    \hline
    j = 5, k>12, \text{sod }  \text{in } v \in L_j^{in}  & [1, 3, 6, 9, 12, 14, 14, 13, 12, 11, 10, 5^{\#(2k - 21)}]  \\
    \hline
    j = 5, k>12, \text{sod }  \text{in } v \in L_j^{out}  & [1, 3, 6, 9, 12, 14, 14, 13, 12, 11, 5^{\#(2k - 19)}]  \\
    \hline
    j = 5, k>13, \text{lih }  \text{in } v \in L_j^{in} &  [1, 3, 6, 9, 12, 14, 14, 13, 12, 11, 10, 5^{\#(2k - 23)}] \\
    \hline
    j = 5, k>13, \text{lih }  \text{in } v \in L_j^{out} & [1, 3, 6, 9, 12, 14, 14, 13, 12, 11, 5^{\#(2k - 21)}] \\
    \hline
    j > 5, k>14, \text{sod }  \text{in } v \in L_j^{out}  & [1, 3, 6, 9, 12, 14, 14, 13, 12, 11, 10^{\#(2j - 10)}, 5^{\#(2k - 4j + 1)}]  \\
    \hline
    j > 5, k>14, \text{sod }  \text{in } v \in L_j^{in}  & [1, 3, 6, 9, 12, 14, 14, 13, 12, 11, 10^{\#(2j - 9)}, 5^{\#(2k - 4j - 1)}] \\
    \hline
    j > 5, k>15, \text{lih }  \text{in } v \in L_j^{out} & [1, 3, 6, 9, 12, 14, 14, 13, 12, 11, 10^{\#(2j - 9)}, 5^{\#(2k - 4j - 3)}] \\
    \hline
    j > 5, k>15, \text{lih }  \text{in } v \in L_j^{in} & [1, 3, 6, 9, 12, 14, 14, 13, 12, 11, 10^{\#(2j - 10)}, 5^{\#(2k - 4j - 1)}] \\
    \hline
    j = \lfloor k/2 \lfloor, k \text{ sod }  \text{ za vsak } v \in L_j  & [1, 3, 6, 9, 12, 14, 14, 13, 12, 11, 10^{\#(k - 10)}, 5]        \\
    \hline
    j = \lfloor k/2 \lfloor, k \text{ lih }  \text{ za vsak } v \in L_j  & [1, 3, 6, 9, 12, 14, 14, 13, 12, 11, 10^{\#(k - 11)}, 5]        \\
    \hline
\end{array}$
\caption{Vewktorji distančne particije za graf $C_{10k}$, kjer je $k \geq 10$. Notaciji $\#5^k$ in $\#10^k$, nam povejo število petic oziroma desetic na koncu distančnega vektorja}
\end{table}

V Tabela 2 lahko opazimo, da distančni vektor ni odvisen zgolj od tega, ali je vozlišče $v$ vhodno ali izhodno, temveč tudi od tega ali je $k$ sodo ali liho število.

\section{Indeksi na podlagi distanc v $(5,0)$ nanotubu}

V tem oddelku bomo testirali različne indekse ki jih izračunamo na podlagi distanc v grafu $C_{10k}$

\begin{definicija}
    Indeks ekscentrične povezanosti za graf $G$ izračunamo tako
    $$
    \xi_{c}(G) = \sum_{v \in V(G)}deg_G(v)ecc_G(v),
    $$
    kjer z $deg_G(v)$ označimo stopnjo vozlišča $v$ v grafu $G$.
\end{definicija}

\begin{definicija}
    Indeks sosednje ekscentričnosti povezanosti za graf $G$ izračunamo tako
    $$
    \xi^{ad}(G) = \sum_{v \in V(G)}\frac{SG(v)}{ecc_G(v)},
    $$
    kjer z $SG(v)$ označimo seštevek stopenj vozlišč, ki so sosenja vozlišču v.
\end{definicija}

\begin{definicija}
    Prvi Zagrebški indeks povezanosti je definiran kot
    $$
    \xi_{1}(G) = \sum_{u \in E(G)}ecc_G^2(u),
    $$
    in drugi Zagrebški indeks povezanosti je definiran kot
    $$
    \xi_{2}(G) = \sum_{uv \in E(G)}(ecc_G(u)ecc_G(v)).
    $$
\end{definicija}

\begin{izrek}
    Naj bo $k \geq 8$. Potem velja:
    \begin{enumerate}
        \item $\xi_{c}(C_{10k}) = 45k^2 - 15k$,
        \item $90 \cdot \ln(2) \leq \xi^{ad}(C_{10k}) \leq 90 \cdot (\ln(2k - 1) - \ln(k - 1))$,
        \item $\xi_{1}(C)_{10k} = 45k^2 - 15k$
        \item $\xi_{2}(C_{10k}) = 35k^3 - \frac{45}{2}k^2 + \delta,$ kjer $\delta = -5k + 10$, če je $k$ sodo in $\delta = 15 / 2 - 5 k$, če je $k$ liho. 
    \end{enumerate}
\end{izrek}

Ta izrek sva preverila in res drži v prvem, drugem in zadnjem primeru, v tretjem pa ne. Prav tako ni nobe očitne rešitve za ta primer. 
Opazila pa sva tudi da druga točka velja za $k \geq 6$.


\begin{izrek}
 Naj bo $k \geq 10$ in $t$ tak, da velja $10 \leq t \leq 2k - 1$. Tedaj velja:
    \begin{align}
        W1(C{10k}) &= 15k, \
        W3(C{10k}) &= 45k - 30, \
        W5(C{10k}) &= 70k - 135, \
        W7(C{10k}) &= 65k - 220, \
        W9(C{10k}) &= 55k - 250, \
        W2(C{10k}) &= 30k, \
        W4(C{10k}) &= 60k - 80, \
        W6(C{10k}) &= 70k - 180, \
        W8(C{10k}) &= 60k - 230, \
        Wt(C{10k}) &= 50k - 25t.
    \end{align}
\end{izrek}


Ugotovitve so sledeče:
\begin{itemize}
    \item Spremenljivka $t$ je vedno navzgor omejena z $2k - 1$, saj je to premer grafa. Zato je smiselno, da velja $W_t = 0$ za $t \geq 2k - 1$.
    \item Trditev je pravilna za vse $k \geq 2$, če le upoštevamo, da je $t$ omejen s premerom grafa.
\end{itemize}



\begin{izrek}
  Naj bo $k \geq 10$. Tedaj velja:
    \begin{align}
        W(C{10k}) &= \frac{100}{3} k^3 + \frac{1175}{3} k - 670, \
        WW(C{10k}) &= \frac{100}{3} k^4 + \frac{100}{3} k^3 - \frac{25}{3} k^2 + \frac{10175}{3} k - 7200, \
        RCW(G) &= R_k + 50k - 250.
    \end{align}
   \end{izrek}
   
Tu je napaka v formuli za Rk.

Pravilna formula je sledeča:

[\(

R_k = (15 * k) / (2 * k - 1) + (30 * k) / (2 * k - 2) + (45 * k - 30) / (2 * k - 3) \
+ (60 * k - 80) / (2 * k - 4) + (70 * k - 135) / (2 * k - 5) \
+ (70 * k - 180) / (2 * k - 6) + (65 * k - 220 ) / (2 * k - 7) \
+ (60 * k - 230) / (2 * k - 8) + (55 * k - 250) / (2 * k - 9)



\)]



   \begin{izrek}


    Naj bo $\alpha \in \mathbb{R} \setminus {0,1}$. Označimo 
    [
    W9^\alpha = \sum{t=1}^{9} t \cdot Wt(C{10k}).
    ]
    Tedaj velja:
    [
    W9^\alpha + L < W^\alpha(C{10k}) < W_9^\alpha + P,
    ]
    kjer sta $L$ in $P$ spodnja in zgornja meja, določeni na sledeč način:


  
[
L =
\begin{cases} 
\frac{50k}{\alpha + 1} \left( (2k)^{\alpha + 1} - 10^{\alpha + 1} \right) - \frac{25}{\alpha + 2} \left( (2k-1)^{\alpha + 2} - 9^{\alpha + 2} \right), & \text{če } \alpha < 0, \alpha \neq -1, -2, \
\frac{50k}{\alpha + 1} \left( (2k-1)^{\alpha + 1} - 9^{\alpha + 1} \right) - \frac{25}{\alpha + 2} \left( (2k-1)^{\alpha + 2} - 9^{\alpha + 2} \right), & \text{če } 0 < \alpha < 1, \
\frac{50k}{\alpha + 1} \left( (2k-1)^{\alpha + 1} - 9^{\alpha + 1} \right) - \frac{25}{\alpha + 2} \left( (2k)^{\alpha + 2} - 10^{\alpha + 2} \right), & \text{če } \alpha > 1, \
50k \left( \ln(2k) - \ln(10) \right), & \text{če } \alpha = -1, \
-50k \left( (2k)^{-1} - 10^{-1} \right) - 25 \left( \ln(2k-1) - \ln(9) \right), & \text{če } \alpha = -2.
\end{cases}
]


[
P =
\begin{cases} 
\frac{50k}{\alpha + 1} \left( (2k-1)^{\alpha + 1} - 9^{\alpha + 1} \right) - \frac{25}{\alpha + 2} \left( (2k)^{\alpha + 2} - 10^{\alpha + 2} \right), & \text{če } \alpha < 0, \alpha \neq -1, -2, \
\frac{50k}{\alpha + 1} \left( (2k)^{\alpha + 1} - 10^{\alpha + 1} \right) - \frac{25}{\alpha + 2} \left( (2k)^{\alpha + 2} - 10^{\alpha + 2} \right), & \text{če } 0 < \alpha < 1, \
\frac{50k}{\alpha + 1} \left( (2k)^{\alpha + 1} - 10^{\alpha + 1} \right) - \frac{25}{\alpha + 2} \left( (2k-1)^{\alpha + 2} - 9^{\alpha + 2} \right), & \text{če } \alpha > 1, \
50k \left( \ln(2k-1) - \ln(9) \right) - 50k + 250, & \text{če } \alpha = -1, \
-50k \left( (2k-1)^{-1} - 9^{-1} \right) - 25 \left( \ln(2k) - \ln(10) \right), & \text{če } \alpha = -2.
\end{cases}
]

   \end{izrek}


   Kljub večkratnim preverjanjem in različnim pristopom nikakor ne uspe priti do pravilnega rezultata, 
   kjer bi izrek veljal.
   
   Zanimivo pa je, da izrek drži za vse $\alpha$, razen za $\alpha = -2$, če odstranimo $W9$ iz ocen:
   [
   L < W^\alpha(C{10k}) < P.
   ]
   To kaže na morebitno napačno vključitev $W_9$ v meje ali potrebo po drugačnem načinu upoštevanja tega člena.


   \begin{table}[h]
    \centering
    \small
    \begin{tabular}{cccccccc}
        \toprule
        $k$ & $\alpha$ & Spodnja meja & $W^\alpha(C_{10k})$ & Zgornja meja & $W_9^\alpha$ & $L$ & $P$ \
        \midrule
        20 & -2.0 & 45243.34 & 44.24 & 45255.81 & 45205 & 38.34 & 50.81 \
        21 & -2.0 & 47897.09 & 48.28 & 47910.18 & 47855 & 42.09 & 55.18 \
        22 & -2.0 & 50550.90 & 52.37 & 50564.60 & 50505 & 45.90 & 59.60 \
        23 & -2.0 & 53204.76 & 56.52 & 53219.07 & 53155 & 49.76 & 64.07 \
        24 & -2.0 & 55858.68 & 60.71 & 55873.59 & 55805 & 53.68 & 68.59 \
        25 & -2.0 & 58512.64 & 64.95 & 58528.14 & 58455 & 57.64 & 73.14 \
        26 & -2.0 & 61166.63 & 69.23 & 61182.74 & 61105 & 61.63 & 77.74 \
        27 & -2.0 & 63820.67 & 73.54 & 63837.37 & 63755 & 65.67 & 82.37 \
        28 & -2.0 & 66474.75 & 77.89 & 66492.03 & 66405 & 69.75 & 87.03 \
        29 & -2.0 & 69128.85 & 82.27 & 69146.73 & 69055 & 73.85 & 91.73 \
        \midrule
        20 & -1.0 & 45921.34 & 674.57 & 46591.29 & 45205 & 1386.29 & 716.34 \
        21 & -1.0 & 48647.16 & 747.66 & 49361.84 & 47855 & 1506.84 & 792.16 \
        22 & -1.0 & 51375.37 & 823.13 & 52134.76 & 50505 & 1629.76 & 870.37 \
        23 & -1.0 & 54105.85 & 900.88 & 54909.96 & 53155 & 1754.96 & 950.85 \
        24 & -1.0 & 56838.51 & 980.79 & 57687.34 & 55805 & 1882.34 & 1033.51 \
        25 & -1.0 & 59573.24 & 1062.80 & 60466.80 & 58455 & 2011.80 & 1118.24 \
        26 & -1.0 & 62309.98 & 1146.80 & 63248.26 & 61105 & 2143.26 & 1204.98 \
        27 & -1.0 & 65048.64 & 1232.72 & 66031.64 & 63755 & 2276.64 & 1293.64 \
        28 & -1.0 & 67789.15 & 1320.50 & 68816.87 & 66405 & 2411.87 & 1384.15 \
        29 & -1.0 & 70531.45 & 1410.07 & 71603.89 & 69055 & 2548.89 & 1476.45 \
        \midrule
        20 & 0.5 & 94747.35 & 50711.74 & 97018.53 & 45205 & 51813.53 & 49542.35 \
        21 & 0.5 & 105094.84 & 58487.30 & 107518.15 & 47855 & 59663.15 & 57239.84 \
        22 & 0.5 & 116090.38 & 66910.92 & 118665.86 & 50505 & 68160.86 & 65585.38 \
        23 & 0.5 & 127749.22 & 75997.84 & 130476.91 & 53155 & 77321.91 & 74594.22 \
        24 & 0.5 & 140086.26 & 85762.98 & 142966.20 & 55805 & 87161.20 & 84281.26 \
        25 & 0.5 & 153116.09 & 96220.92 & 156148.33 & 58455 & 97693.33 & 94661.09 \
        26 & 0.5 & 166853.01 & 107385.94 & 170037.58 & 61105 & 108932.58 & 105748.01 \
        27 & 0.5 & 181311.02 & 119272.06 & 184647.95 & 63755 & 120892.95 & 117556.02 \
        28 & 0.5 & 196503.85 & 131893.01 & 199993.18 & 66405 & 133588.18 & 130098.85 \
        29 & 0.5 & 212444.99 & 145262.28 & 216086.76 & 69055 & 147031.76 & 143389.99 \
        \bottomrule
    \end{tabular}
    \caption{Primeri, kjer izrek ne drži.}
\end{table}


Na zagovoru bodo predstavljeni tudi različni indeksi in njihove lastnosti, saj je prikaz s pomočjo grafov še toliko bolj zanimiv.



\end{document}