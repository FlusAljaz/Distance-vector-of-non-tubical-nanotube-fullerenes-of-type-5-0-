\documentclass[a4paper, 12pt]{article}
\usepackage[slovene]{babel}
\usepackage[utf8]{inputenc}
\usepackage[T1]{fontenc}
\usepackage{lmodern}
\usepackage{amsmath, amssymb}

\newtheorem{izrek}{Izrek}[section]

%\theoremstyle{definition}
\newtheorem{definicija}{Definicija}[section]

\newtheorem{opomba}{Opomba}[section]

\title{
    Distance vector of non tubical nanotube fullerenes of type-(5-0)
}

\author{Amanda Babič, Aljaž Flus \\
        {\small Mentorja Riste Škrekovski, Janoš Vidali}}
\date{19. 12. 2024}

\begin{document}

\maketitle
\section{Uvod}

Nanostrukture, kot so fulereni, predstavljajo eno najbolj zanimivih odkritij na področju nanotehnologije in kemije. Med temi strukturami so še posebej zanimivi fulereni tipa $(5,0)$, ki zaradi svoje posebne geometrije in ostalih lastnosti omogočajo vpogled v kompleksne topološke in matematične strukture.  
Ti fulereni so definirani z $n=10k$ vozlišči in imajo premer $\frac{n}{5}-1$. 
Njihova geometrija vključuje dolg cilindrični del, sestavljen iz šestkotnikov, ki je na obeh straneh zaključen s šestimi petkotniki.  
Ti grafi so simetrični, kar bomo uporabili v nadaljevanju naloge. Razumevanje razdalj med točkami v teh grafih je ključno za odkrivanje njihovih lastnosti. V tem okviru so vozlišča grafov razporejena v $k$ ločenih orbitah, pri čemer vsaka orbita vsebuje točno 10 vozlišč.  \\
Glavni cilj te naloge je čim bolj natančno izračunati vektor razdalj za izbrano vozlišče iz vsake orbite.

\section{Definicije}

\begin{definicija}
    Vektor razdalje $d_{u}$ je vektor, katerega $i$-ta koordinata predstavlja število vozlišč, ki so od izbranega vozlišča $u$ oddaljeni natanko za $i$. 
\end{definicija}

\begin{definicija}
    Graf fulerena je 3-povezan, 3-regularen ravninski graf, sestavljen izključno iz petkotnih in šestkotnih ploskev.
\end{definicija}

\begin{opomba}
    Po Eulerjevi formuli je število petkotnih ploskev vedno 12.
\end{opomba}

\begin{definicija}
    Nanocevke so cilindrični fulerenski grafi, pri katerih sta oba konca cilindra zaprta s podgrafoma, ki sta sestavljena iz šestih petkotnikov.
\end{definicija}

Cilindrični del nanocevke je določen s $(p_1, p_2)$-vektorjem, ki opisuje način ovijanja neskončne mreže šestkotnikov, da nastane cilindrična struktura.  
Števili $p_1$ in $p_2$ označujeta koeficiente linearne kombinacije enotskih vektorjev $a_1$ in $a_2$, pri čemer $p_1a_1 + p_2a_2$ povezuje isto točko na mreži, ko jo zavijemo v cilindrično obliko.

\begin{opomba}
    Predpostavimo, da velja $p_1 \geq p_2$, s čimer se izognemo simetričnemu opisu iste konfiguracije.
\end{opomba}


\begin{definicija} 
    Wienerjeva dimenzija grafa je število različnih vrednosti razdalj med pari vozlišč v grafu. 
\end{definicija} 

\begin{opomba} 
    Za graf tipa $(5,0)$ z $n = 10k$ vozlišči velja, da je Wienerjeva dimenzija enaka $k$, če je $k \geq 3$. 
\end{opomba} 


Zaradi simetrije grafa tipa $(5,0)$ je mogoče predvideti, da se vozlišča razporejajo v določene "razdaljne razrede". 
Wienerjeva dimenzija to potrjuje, saj zbere vse vozlišča z enakimi razdaljami v iste razrede. 

To poenostavi analizo razdalj v grafu. 
Namesto da za vsak par vozlišč obravnavamo njihove razdalje posamično, lahko uporabimo Wienerjevo dimenzijo za njihovo kategorizacijo.

\section{Načrt dela}

Nalogo bova začela z implementacijo Floyd-Warshallovega algoritma za izračun najkrajših razdalj med vsemi pari vozlišč v grafu.
Na podlagi izračunanih razdalj bova za vsako vozlišče skonstruirala vektor dolžin, ki bo vseboval razdalje do vseh drugih vozlišč.
Nato bova analizirala te vektorje, da bi identificirala zanimive lastnosti ali vzorce, ki jih lahko izluščiva iz teh podatkov.

\end{document}
